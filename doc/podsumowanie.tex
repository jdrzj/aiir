\documentclass[a4paper,10pt]{article}
\usepackage[top=1.5in, bottom=1.5in, left=0.8in, right=0.8in]{geometry}
\usepackage[utf8x]{inputenc}
\usepackage[T1]{fontenc}
\usepackage[english,polish]{babel}
\usepackage{titlesec}
\setcounter{secnumdepth}{4}
\titleformat{\paragraph}
{\normalfont\normalsize\bfseries}{\theparagraph}{1em}{}
\titlespacing*{\paragraph}
{0pt}{3.25ex plus 1ex minus .2ex}{1.5ex plus .2ex}
\usepackage{indentfirst}
\usepackage{hhline}
\usepackage{fixltx2e}
\usepackage{hyperref}
\usepackage{graphicx}
\usepackage{listings}
\usepackage{capt-of}
\usepackage{tabularx}
\usepackage{multirow}
\usepackage{array}
\usepackage{listingsutf8}
\usepackage{longtable}
\usepackage{tikz}
\usepackage{tabularx}
\usepackage{float}
\usetikzlibrary{matrix,calc}
\usepackage{listings}
\lstset{
    language=C,
    basicstyle=\ttfamily\small,
    backgroundcolor=\color{gray!10},
    frame=single,
    tabsize=2,
    rulecolor=\color{black!30},
    title=\lstname,
    escapeinside={\%*}{*)},
    breaklines=\true,
    breakatwhitespace=true,
    framextopmargin=2pt,
    framexbottommargin=2pt,
    inputencoding=utf8x, 
    extendedchars=\true,
    literate={ą}{{\k{a}}}1
             {Ą}{{\k{A}}}1
             {ę}{{\k{e}}}1
             {Ę}{{\k{E}}}1
             {ó}{{\'o}}1
             {Ó}{{\'O}}1
             {ś}{{\'s}}1
             {Ś}{{\'S}}1
             {ł}{{\l{}}}1
             {Ł}{{\L{}}}1
             {ż}{{\.z}}1
             {Ż}{{\.Z}}1
             {ź}{{\'z}}1
             {Ź}{{\'Z}}1
             {ć}{{\'c}}1
             {Ć}{{\'C}}1
             {ń}{{\'n}}1
             {Ń}{{\'N}}1
}
\renewcommand{\topfraction}{.85}
\renewcommand{\bottomfraction}{.7}
\renewcommand{\textfraction}{.15}
\renewcommand{\floatpagefraction}{.66}
\renewcommand{\dbltopfraction}{.66}
\renewcommand{\dblfloatpagefraction}{.66}
\addto\captionspolish{
  \renewcommand{\contentsname}
    {Spis treści}
}

\begin{document}

\thispagestyle{empty}


\hfill Wrocław, dn.\today\\
\begin{minipage}[c]{0.4\columnwidth}
  \textbf{Zespół:}\\
  Piotr Chmiel\\
  Tomasz Bagiński\\
  Maciej Stelmaszuk\\
  Jakub Stasiak\\
  Jędrzej Urbański\\
  Aleksandra Berjak\\
  
\end{minipage}
\hfill
\vspace{2.5cm}
\begin{center}
  \begin{LARGE}
    \emph{ Aplikacje internetowe i rozproszone - projekt} \\
  \end{LARGE}
  \begin{Large}
    \emph{Podsumowanie dotychczasowych osiągnięć} \\
  \end{Large}
  \vspace{1.0cm}
    Sprawdzenie czy zadany zbiór zaszyfrowanych haseł może (i w jakim czasie) zostać złamany za pomocą komputera równoległego jakim jest klaster stacji roboczych. Różne metody łamania haseł (słownikowa, brute force, tablice tęczowe)
\end{center}

\begin{center}
  Rok akad. 2014/2015, kierunek: INF
\end{center}
\vspace{1.2cm}
\begin{flushright}
\begin{minipage}[t]{0.4\columnwidth}
\noindent
PROWADZĄCY:\\
dr hab. inż.~Henryk Maciejewski
\end{minipage}
\end{flushright}
\vfill
\newpage
\tableofcontents{}
\section{Wykonane zadania: Ogólnie}

\begin{enumerate}
	\item Implementacja algorytmu łamania hasła: metoda słownikowa dla hashy SHA1. (wersja MPI)
	\item Implementacja algorytmu łamania hasła: tablice tęczowe dla hashy SHA1, hasła numeryczne. (wersja MPI)
	\item Implementacja algorytmu łamania hasła: metoda brute force dla hashy SHA1. (wersja sekwencyjna)
	\item Opracowanie i implementacja komunikacji pomiędzy aplikacją MPI, a aplikacją webową Django.
	\item Implementacja aplikacji klienckiej Django.
	
\end{enumerate}

\section{Wykonane zadania: Jędrzej Urbański}

\subsection{Implementacja algorytmu łamania hasła: metoda słownikowa dla hashy SHA1. (wersja MPI)}

\begin{enumerate}
	\item Implementacja algorytmu słownikowego wraz z modułem do wszystkich możliwych wariacji hasła i dopisywanie możliwości przyrostków.
	\item Połączenie modułu C++ i aplikacji klienckiej Django z wykorzystaniem dostępnych modułów.
\end{enumerate}


\section{Wykonane zadania: Aleksandra Berjak}

\subsection{Implementacja algorytmu łamania hasła: tablice tęczowe dla hashy SHA1, hasła numeryczne. (wersja MPI)}

\begin{enumerate}
	\item Stworzenie klasy typu utility jako narzędzie do operacji na hashach (reprezentacja hasha w pamięci, porównywanie dwóch hashy, konwersja hasha z postaci łańcuchu znaków heksadecymalnych do tablicy o określonej liczbie bajtów (np. 20 dla SHA1).
	\item Implementacji algorytmu przeszukiwania tablic.
	\item Implementacja prostego generatora tablic dla ciągów składających się wyłącznie z cyfr 0-9.
	\item Możliwość podzielenia zadania dzielenia hasła na mniejsze podzadania.
\end{enumerate}

\section{Wykonane zadania: Tomasz Bagiński}

\subsection{Implementacja algorytmu łamania hasła: metoda brute force dla hashy SHA1. (wersja sekwencyjna)}

\begin{enumerate}
	\item Implementacja funkcji generującej ciągi znaków. które następnie są hashowane i porównywane z hashem wygenerowanym (za pomocą biblioteki OpenSSL) dla łamanego hasła.
	\item Implementacja funkcji inkrementującej ciąg znaków do generowania potencjalnych haseł. 
	
\end{enumerate}

\section{Wykonane zadania: Jakub Stasiak}

\subsection{Opracowanie i implementacja komunikacji pomiędzy aplikacją MPI, a aplikacją webową Django.}

\begin{enumerate}
	\item Opracowanie schematu bazy danych do przechowywania zadań.
	\item Stworzenie plików konfiguracyjnych Vagrant dla klastra.
	\item Opracowanie i implementacja architektury komunikacyjnej wraz z jej opisem.
\end{enumerate}

\section{Wykonane zadania: Piotr Chmiel}

\subsection{Implementacja aplikacji klienckiej Django}

\begin{enumerate}
	\item Implementacja logowania, wylogowywania, zmiany hasła i autoryzacji użytkowników.
	\item Implementacja panelu administratora.
	\item Implementacja widoku pokazującego aktualne zadania w tabeli.
	\item Implementacja widoku pokazującego historię zadań w tabeli.
	\item Implementacja widoku szczegółówego zadania (pasek postępu).
\end{enumerate}

\section{Wykonane zadania: Maciej Stelmaszuk}

\subsection{Implementacja aplikacji klienckiej Django}

\begin{enumerate}
	\item Implementacja rejestracji użytkowników.
	\item Implementacja widoku edycji profilu użytkownika.
	\item Implementacja formularza dodawania pojedynczego zadania.
	\item Implementacja formularza dodawania wielu zadań z pliku *.txt.
\end{enumerate}

\end{document}