\documentclass[a4paper,10pt]{article}
\usepackage[top=1.5in, bottom=1.5in, left=0.8in, right=0.8in]{geometry}
\usepackage[utf8x]{inputenc}
\usepackage[T1]{fontenc}
\usepackage[english,polish]{babel}
\usepackage{titlesec}
\setcounter{secnumdepth}{4}
\titleformat{\paragraph}
{\normalfont\normalsize\bfseries}{\theparagraph}{1em}{}
\titlespacing*{\paragraph}
{0pt}{3.25ex plus 1ex minus .2ex}{1.5ex plus .2ex}
\usepackage{indentfirst}
\usepackage{hhline}
\usepackage{fixltx2e}
\usepackage{hyperref}
\usepackage{graphicx}
\usepackage{listings}
\usepackage{capt-of}
\usepackage{tabularx}
\usepackage{multirow}
\usepackage{array}
\usepackage{listingsutf8}
\usepackage{longtable}
\usepackage{tikz}
\usepackage{tabularx}
\usepackage{float}
\usetikzlibrary{matrix,calc}
\usepackage{listings}
\lstset{
    language=C,
    basicstyle=\ttfamily\small,
    backgroundcolor=\color{gray!10},
    frame=single,
    tabsize=2,
    rulecolor=\color{black!30},
    title=\lstname,
    escapeinside={\%*}{*)},
    breaklines=\true,
    breakatwhitespace=true,
    framextopmargin=2pt,
    framexbottommargin=2pt,
    inputencoding=utf8x, 
    extendedchars=\true,
    literate={ą}{{\k{a}}}1
             {Ą}{{\k{A}}}1
             {ę}{{\k{e}}}1
             {Ę}{{\k{E}}}1
             {ó}{{\'o}}1
             {Ó}{{\'O}}1
             {ś}{{\'s}}1
             {Ś}{{\'S}}1
             {ł}{{\l{}}}1
             {Ł}{{\L{}}}1
             {ż}{{\.z}}1
             {Ż}{{\.Z}}1
             {ź}{{\'z}}1
             {Ź}{{\'Z}}1
             {ć}{{\'c}}1
             {Ć}{{\'C}}1
             {ń}{{\'n}}1
             {Ń}{{\'N}}1
}
\renewcommand{\topfraction}{.85}
\renewcommand{\bottomfraction}{.7}
\renewcommand{\textfraction}{.15}
\renewcommand{\floatpagefraction}{.66}
\renewcommand{\dbltopfraction}{.66}
\renewcommand{\dblfloatpagefraction}{.66}
\addto\captionspolish{
  \renewcommand{\contentsname}
    {Spis treści}
}

\begin{document}

\thispagestyle{empty}


\hfill Wrocław, dn.\today\\
\begin{minipage}[c]{0.4\columnwidth}
  \textbf{Zespół:}\\
  Piotr Chmiel\\
  Tomasz Bagiński\\
  Maciej Stelmaszuk\\
  Jakub Stasiak\\
  Jędrzej Urbański\\
  Aleksandra Berjak\\
  
\end{minipage}
\hfill
\vspace{2.5cm}
\begin{center}
  \begin{LARGE}
    \emph{ Aplikacje internetowe i rozproszone - projekt} \\
  \end{LARGE}
  \begin{Large}
    \emph{ Analiza wymagań} \\
  \end{Large}
  \vspace{1.0cm}
    Sprawdzenie czy zadany zbiór zaszyfrowanych haseł może (i w jakim czasie) zostać złamany za pomocą komputera równoległego jakim jest klaster stacji roboczych. Różne metody łamania haseł (słownikowa, brute force, tablice tęczowe)
\end{center}

\begin{center}
  Rok akad. 2014/2015, kierunek: INF
\end{center}
\vspace{1.2cm}
\begin{flushright}
\begin{minipage}[t]{0.4\columnwidth}
\noindent
PROWADZĄCY:\\
dr inż.~Henryk Maciejewski
\end{minipage}
\end{flushright}
\vfill
\newpage
\tableofcontents{}
\section{Wstęp}
\subsection{Cel}
Celem tego dokumentu jest stworzenie specyfikacji wymagań dotyczących projektu realizowanego w ramach zajęć z przedmiotu  ``Aplikacje internetowe i rozproszone", która ma być podstawą do dalszej pracy w ramach projektu. Dokument pozwala też na weryfikację, stwierdzającą, czy wykonany system spełnia postawione wymagania.
\subsection{Zakres}
W ramach projektu zrealizowany zostanie system udostępniający usługę zlecenia zadania obliczeniowego za pomocą przeglądarki internetowej. Zadanie obliczeniowe dotyczyć będzie sprawdzenia, czy zadany zbiór zaszyfrowanych haseł może (i w jakim czasie) zostać złamany za pomocą komputera równoległego - klastra stacji roboczych. W skład systemu wchodzić będą następujące komponenty:
\begin{enumerate}
\item Aplikacja kliencka do zlecania zadań przez użytkowników
\item Serwer aplikacji
\item Aplikacja równoległa wykonująca obliczenia
\end{enumerate}
Aplikacja kliencka powinna udostępniać użytkownikowi możliwość przekazywania do aplikacji parametrów dotyczących zlecanego zadania, takich jak na przykład zbiór haseł, wybrana metoda rozwiązywania problemu (np. tablice tęczowe, słownikowa, brute force) czy też maksymalny czas oczekiwania na otrzymanie rozwiązania. Celem systemu jest odnalezienie rozwiązania w czasie krótszym bądź równym estymowanemu czasowi znajdowania rozwiązania metodą brute force (obliczanego na podstawie długości hasła, wykorzystanego zbioru znaków oraz możliwości obliczeniowych klastra).
\subsection{Przegląd dokumentu}
Sekcja ``Ogólny opis projektu" zawiera informacje dotyczące przydatności projektu, możliwości oraz ograniczeń projektowanego systemu, także pod kątem środowiska operacyjnego, charakterystykę jego użytkowników oraz założenia i zależności.
W sekcji ``Wymagania" wyspecyfikowane zostały wymagania funkcjonalne oraz niefunkcjonalne dotyczące projektowanego systemu.
\section{Ogólny opis projektu}
\subsection{Przydatność projektowanego systemu i jego możliwości}
Ze względu na ograniczone ramy czasowe projektu, zaprojektowana i stworzona przez nas implementacja systemu umożliwiającego zlecanie zadań obliczeniowych w komputerze równoległym będzie mniej zaawansowana niż inne dostępne na rynku narzędzia. Zakres projektu obejmuje zarówno aplikację kliencką dostępną przez przeglądarkę internetową, a także serwer oraz aplikację równoległą, dlatego wszystkie one zostały opisane w tym dokumencie. Poniższa listy zawiera funkcjonalności, które oferował będzie projektowany system.
\begin{enumerate}
\item Zlecanie zadania obliczeniowego dotyczącego łamania hasła
\item Rejestracja użytkowników systemu
\item Uwierzytelnianie i autoryzacja użytkowników systemu
\item Tworzenie grup użytkowników z określonym poziomem uprawnień
\item Dodanie hasła/zbioru haseł do złamania w postaci pliku lub wprowadzenie go poprzez dostępny w aplikacji klienckiej formularz
\item Prowadzenie metryk dotyczących zużycia procesora i pamięci na stacjach roboczych dostępnych dla użytkowników o określonych poziomach uprawnień
\item Możliwość śledzenia przez użytkowników historii zlecanych zadań obliczeniowych
\item Możliwość zarządzania przez użytkownika zleconymi przez siebie zadaniami obliczeniowymi - edycja parametrów, usuwanie zadań
\item Monitorowanie przez użytkownika postępu zleconego przez niego zadania poprzez tzw. ``pasek postępu"
\item Powiadomienie użytkownika o zakończeniu zadania obliczeniowego za pomocą wiadomości e-mail
\item Dołączanie do każdego zleconego zadania wyników obliczeń takich jak np. uzyskane hasło, czas wykonywania obliczeń
\end{enumerate}
\subsection{Ograniczenia}
Klaster obliczeniowy stanowić będzie grupa stacji roboczych pracujących w środowisku MPI. Realizacja projektu ma charakter iteracyjny. Około 60\% czasu przeznaczonego na projekt (iteracja 0) powinno zostać wykorzystane na stworzenie prototypu umożliwiającego weryfikację zaproponowanej architektury. W iteracji nr 1 rozwiązanie to zostanie rozbudowane. Technologie proponowane przy realizacji projeku to:
\begin{enumerate}
\item Serwer aplikacji: Python, Django/Flask
\item Aplikacja równoległa: MPI (implementacja MPICH, C/C++, ew. Python),
\item Strona klienta: HTML5, CSS, JavaScript (+ jQuery)
\end{enumerate}
\subsection{Charakterystyka użytkowników}
Użytkowników systemu stanowić będą studenci oraz kadra naukowa Politechniki Wrocławskiej. Nie jest planowane rozpowszechnianie systemu poza środowisko akademickie.
\subsection{Środowisko operacyjne}
Stacje robocze będą realizować swoją rolę jako część klastra obliczeniowego przy użyciu wirtualnego środowiska skonfigurowanego specjalnie w tym celu.
\subsection{Założenia i zależności}
W ramach projektu podejmowane będą próby złamania haseł, które zostały przetworzone przy pomocy funkcji haszujących SHA-1 i MD5. Testy prototypu będą wykonywane na komputerach będących własnością grupy projektowej przy założeniu, że uda się poprawnie skonfigurować na nich środowisko. Celem weryfikacji, czy odnalezione przez system rozwiązanie jest poprawne, a hasło zostało złamane, system będzie znał faktyczne hasło, jednak jego znajomość nie będzie miała wpływu na wykonywane obliczenia. Komputery znajdujące się w klastrze mogą charakteryzować się inną specyfikację sprzętową i mieć różne możliwości obliczeniowe.
\section{Wymagania}
\subsection{Wymagania funkcjonalne (funkcje systemu)}
\subsubsection{Aplikacja kliencka}
\begin{table}[H]
\caption{Zlecenie zadania obliczeniowego}
\begin{tabularx}{\textwidth}{ |l|X| }
\hline
ID & FUN\textunderscore KLI\textunderscore 1 \\
\hline
Opis & Aplikacja pozwala użytkownikowi na zlecenia zadania obliczeniowego dotyczącego łamania hasła lub zbioru haseł. \\
\hline
\end{tabularx}
\end{table}
\begin{table}[H]
\caption{Wyświetlanie paska postępu}
\begin{tabularx}{\textwidth}{ |l|X| }
\hline
ID & FUN\textunderscore KLI\textunderscore 2 \\
\hline
Opis & Aplikacja wyświetla pasek postępu zadania obliczeniowego wykonywane przez komputer równoległy.\\
\hline
\end{tabularx}
\end{table}
\begin{table}[H]
\caption{Wprowadzenie hasła lub zbioru haseł w postaci oryginalnej i po działaniu algorytmu haszującego}
\begin{tabularx}{\textwidth}{ |l|X| }
\hline
ID & FUN\textunderscore KLI\textunderscore 3 \\
\hline
Opis & Użytkownik aplikacji ma możliwość podania listy haseł w formie pliku tekstowego lub też wprowadzenie ich w dostępnym w aplikacji klienckiej polu edycyjnym. Separator oddzielający poszczególne hasła powinien być jasno określony w formularzu dotyczącym zlecanego zadania obliczeniowego.\\
\hline
\end{tabularx}
\end{table}
\begin{table}[H]
\caption{Określenie algorytmu haszującego}
\begin{tabularx}{\textwidth}{ |l|X| }
\hline
ID & FUN\textunderscore KLI\textunderscore 4 \\
\hline
Opis & Przed zleceniem zadania obliczeniowego, system prosi użytkownika o podanie algorytmu haszującego, jaki został wykorzystany przy haszowaniu haseł, które mają zostać poddane próbie złamania. Na jego podstawie sugeruje metodę rozwiązania problemu. \\
\hline
\end{tabularx}
\end{table}
\begin{table}[H]
\caption{Oszacowanie czasu łamania hasła metodą brute force}
\begin{tabularx}{\textwidth}{ |l|X| }
\hline
ID & FUN\textunderscore KLI\textunderscore 5 \\
\hline
Opis & Na podstawie wprowadzonego poprawnego hasła (ilość znaków, zakres znaków, możliwości obliczeniowe klastra) system oblicza szacunkowy czas znajdowania odpowiedniego rozwiązania metodą brute force. \\
\hline
\end{tabularx}
\end{table}
\begin{table}[H]
\caption{Śledzenie historii zleconych zadań obliczeniowych}
\begin{tabularx}{\textwidth}{ |l|X| }
\hline
ID & FUN\textunderscore KLI\textunderscore 6 \\
\hline
Opis & Użytkownik aplikacji klienckiej ma możliwość śledzenia historii zleconych przez siebie zadań obliczeniowych, czasu ich wykonywania, ich wyników.\\
\hline
\end{tabularx}
\end{table}
\begin{table}[H]
\caption{Zarządzanie zleconymi zadaniami}
\begin{tabularx}{\textwidth}{ |l|X| }
\hline
ID & FUN\textunderscore KLI\textunderscore 7 \\
\hline
Opis & Użytkownik ma możliwość zarządzania parametrami zleconych przez siebie zadań oraz ich usuwania. Edycja zadania może mieć miejsce przed rozpoczęciem jego wykonywania. Zatrzymanie wykonywania zadania można zlecić zarówno przed jak i w trakcie jego wykonywania. \\
\hline
\end{tabularx}
\end{table}
\begin{table}[H]
\caption{Uwierzytelnianie i autoryzacja użytkowników systemu}
\begin{tabularx}{\textwidth}{ |l|X| }
\hline
ID & FUN\textunderscore KLI\textunderscore 8 \\
\hline
Opis & Użytkownicy systemu logują się do systemu z wykorzystaniem unikalnego loginu i hasła.  \\
\hline
\end{tabularx}
\end{table}
\begin{table}[H]
\caption{Rejestracja użytkowników}
\begin{tabularx}{\textwidth}{ |l|X| }
\hline
ID & FUN\textunderscore KLI\textunderscore 9 \\
\hline
Opis & Użytkownika ma możliwość korzystania z aplikacji klienckiej tylko po stworzeniu konta użytkownika powiązanego z konkretnym adresem e-mail.\\
\hline
\end{tabularx}
\end{table}
\begin{table}[H]
\caption{Walidacja wprowadzanych danych}
\begin{tabularx}{\textwidth}{ |l|X| }
\hline
ID & FUN\textunderscore KLI\textunderscore 10 \\
\hline
Opis & Aplikacja kliencka waliduje dane wprowadzone przez użytkownika, odrzuca hasła lub zbiory haseł nie spełniające określonego schematu.\\
\hline
\end{tabularx}
\end{table}
\subsubsection{Serwer aplikacji}
\begin{table}[H]
\caption{Agregacja metryk dotyczących zużycia zasobów komputerów obliczeniowych}
\begin{tabularx}{\textwidth}{ |l|X| }
\hline
ID & FUN\textunderscore SER\textunderscore 1 \\
\hline
Opis & Serwer udostępnia statystyki dotyczące zużycia procesora i pamięci na komputerach wchodzących w skład klastra użytkownikom o poziomie uprawnień administratora.  \\
\hline
\end{tabularx}
\end{table}
\begin{table}[H]
\caption{Tworzenie grup użytkowników o określonych poziomach uprawnień}
\begin{tabularx}{\textwidth}{ |l|X| }
\hline
ID & FUN\textunderscore SER\textunderscore 2 \\
\hline
Opis & Przynależność użytkownika do konkretnych grup/y określa jego prawo do korzystania z konkretnych zasobów systemu\\
\hline
\end{tabularx}
\end{table}
\begin{table}[H]
\caption{Monitorowanie zużycia zasobów przez komputery należące do klastra}
\begin{tabularx}{\textwidth}{ |l|X| }
\hline
ID & FUN\textunderscore SER\textunderscore 3 \\
\hline
Opis & Serwer monitoruje poziom zużycia zasobów takich jak procent zużycia CPU na komputerach należących do klastra. W rozbudownanej wersji architektury może zostać zaimplementowana funkcjonalność przekierowania części obliczeń z komputera, na którym zużycie CPU osiągnęło zbyt wysoki poziom. \\
\hline
\end{tabularx}
\end{table}
\begin{table}[H]
\caption{Zlecanie zadań obliczeniowych}
\begin{tabularx}{\textwidth}{ |l|X| }
\hline
ID & FUN\textunderscore SER\textunderscore 4 \\
\hline
Opis & System zleca węzłom rozwiązanie określonego zadania obliczeniowego za pomocą protokołu ustalonego w toku pracy nad projektem.\\
\hline
\end{tabularx}
\end{table}
\begin{table}[H]
\caption{Monitorowanie stanu wykonywania zadania}
\begin{tabularx}{\textwidth}{ |l|X| }
\hline
ID & FUN\textunderscore SER\textunderscore 5 \\
\hline
Opis & Serwer komunikuje się z węzłami i na podstawie otrzymanych danych określa poziom wykonywania zadania obliczeniowego, który udostępnia aplikacji klienckiej.\\
\hline
\end{tabularx}
\end{table}
\begin{table}[H]
\caption{Zlecenie węzłowi zadania na podstawie jego mocy obliczeniowej}
\begin{tabularx}{\textwidth}{ |l|X| }
\hline
ID & FUN\textunderscore SER\textunderscore 6 \\
\hline
Opis & Na podstawie danych dotyczących specyfikacji komputera oraz wyników prostych testów, serwer ustala współczynnik mocy obliczeniowej komputera w stosunku do całego klastra i na tej podstawie określa, jak duże zadanie obliczeniowe może mu przydzielić, by zrównoważyć czas jego wykonywania na wszystkich stacjach roboczych.\\
\hline
\end{tabularx}
\end{table}
\begin{table}[H]
\caption{Informowanie użytkownika o zakońćzeniu zadania obliczeniowego}
\begin{tabularx}{\textwidth}{ |l|X| }
\hline
ID & FUN\textunderscore SER\textunderscore 7 \\
\hline
Opis & Po zakończeniu wykonywania zadania obliczeniowego, serwer wysyła użytkownikowi zlecającemu je wiadomość e-mail informującą o wynikach obliczeń.\\
\hline
\end{tabularx}
\end{table}
\subsubsection{Aplikacja równoległa}
\begin{table}[H]
\caption{Wykonywanie zadania obliczeniowego}
\begin{tabularx}{\textwidth}{ |l|X| }
\hline
ID & FUN\textunderscore WEZ\textunderscore 1 \\
\hline
Opis & Aplikacja równoległa wykonuje zlecone jej przez serwer zadanie obliczeniowe.\\
\hline
\end{tabularx}
\end{table}
\begin{table}[H]
\caption{Raportowanie postępu w wykonywaniu zadania}
\begin{tabularx}{\textwidth}{ |l|X| }
\hline
ID & FUN\textunderscore WEZ\textunderscore 2 \\
\hline
Opis & Aplikacji kliencka wysyła serwerowi informacje dotyczące postępu wykonywania zadania w odstępach jednego procenta.\\
\hline
\end{tabularx}
\end{table}
\begin{table}[H]
\caption{Udostępnianie danych dotyczących zużycia zasobów}
\begin{tabularx}{\textwidth}{ |l|X| }
\hline
ID & FUN\textunderscore WEZ\textunderscore 3 \\
\hline
Opis & Na żądanie serwera aplikacja udostępnia dane dotyczące aktualnego zużycia zasobów na danej stacji roboczej takich jak zużycie CPU czy pamięci.\\
\hline
\end{tabularx}
\end{table}
\subsection{Wymagania niefunkcjonalne (ograniczenia)}
\begin{table}[H]
\caption{Interfejs użytkownika aplikacji klienckiej}
\begin{tabularx}{\textwidth}{ |l|X| }
\hline
ID & NONFUN\textunderscore 1 \\
\hline
Opis & Aplikacja kliencka powinna posiadać graficzny interfejs użytkownika, być przejrzysta, intuicyjna i łatwa w obsłudze.\\
\hline
\end{tabularx}
\end{table}
\begin{table}[H]
\caption{Język aplikacji}
\begin{tabularx}{\textwidth}{ |l|X| }
\hline
ID & NONFUN\textunderscore 2 \\
\hline
Opis & Interfejs aplikacji klienckiej powinien być dostępny w języku polskim lub angielskim.\\
\hline
\end{tabularx}
\end{table}
\begin{table}[H]
\caption{Przenośność aplikacji równoległych}
\begin{tabularx}{\textwidth}{ |l|X| }
\hline
ID & NONFUN\textunderscore 3 \\
\hline
Opis & Aplikacje równoległe powinny mieć możliwość łatwej adaptacji do różnych systemów operacyjnych.\\
\hline
\end{tabularx}
\end{table}
\begin{table}[H]
\caption{Skalowalność}
\begin{tabularx}{\textwidth}{ |l|X| }
\hline
ID & NONFUN\textunderscore 3 \\
\hline
Opis & System powinien być skalowalny, należy uwzględnić możliwość zwiększenia ilości stacji roboczych nawet dziesięciokrotnie.\\
\hline
\end{tabularx}
\end{table}
\begin{table}[H]
\caption{Zdolność adaptacji do możliwości danej stacji roboczej}
\begin{tabularx}{\textwidth}{ |l|X| }
\hline
ID & NONFUN\textunderscore 3 \\
\hline
Opis & System powinien indywidualnie rozpatrywać możliwości obliczeniowe każdej ze stacji roboczych i na ich podstawie określać zlecany jej nakład pracy.\\
\hline
\end{tabularx}
\end{table}

\end{document}